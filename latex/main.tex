\documentclass[12pt, oneside]{book}
\usepackage[a4paper,top=2.5cm,bottom=2.5cm,left=3.5cm,right=2cm]{geometry}
\usepackage[utf8]{inputenc}
\usepackage[T1]{fontenc}
\usepackage{graphicx}
\usepackage{url}
\usepackage{subcaption}
\usepackage[hidelinks,breaklinks]{hyperref}
\usepackage[slovak]{babel} % vypnite pre prace v anglictine
\usepackage{float}
\usepackage{listings}
\usepackage{csquotes}
\usepackage{mathtools}
\usepackage{amsfonts} 
\usepackage[table]{xcolor}
\usepackage{amsmath}
\usepackage{algorithm}
\usepackage{algpseudocode}
\usepackage{cmap}
\usepackage{lmodern} 

\usepackage{amssymb}% http://ctan.org/pkg/amssymb
\usepackage{pifont}% http://ctan.org/pkg/pifont
\newcommand{\cmark}{\ding{51}}%
\newcommand{\xmark}{\ding{55}}%

\usepackage{array}
\newcolumntype{L}[1]{>{\raggedright\let\newline\\\arraybackslash\hspace{0pt}}m{#1}}
\usepackage{array}
\newcolumntype{C}[1]{>{\centering\let\newline\\\arraybackslash\hspace{0pt}}m{#1}}

\DeclareMathAlphabet{\mathcal}{OMS}{cmsy}{m}{n}
\SetMathAlphabet{\mathcal}{bold}{OMS}{cmsy}{b}{n}
\linespread{1.25} % hodnota 1.25 by mala zodpovedat 1.5 riadkovaniu

\algnewcommand\algorithmicforeach{\textbf{for each}}
\algdef{S}[FOR]{ForEach}[1]{\algorithmicforeach\ #1\ \algorithmicdo}

\usepackage{color}

\definecolor{dkgreen}{rgb}{0,0.6,0}
\definecolor{gray}{rgb}{0.5,0.5,0.5}
\definecolor{mauve}{rgb}{0.58,0,0.82}

\lstset{frame=tb,
  language=Java,
  aboveskip=3mm,
  belowskip=3mm,
  showstringspaces=false,
  columns=flexible,
  basicstyle={\footnotesize\ttfamily},
  numbers=none,
  numberstyle=\tiny\color{gray},
  keywordstyle=\color{blue},
  commentstyle=\color{dkgreen},
  stringstyle=\color{mauve},
  breaklines=true,
  breakatwhitespace=true,
  tabsize=3
}


% -------------------
% --- Definicia zakladnych pojmov
% --- Vyplnte podla vasho zadania
% -------------------
\def\mfrok{2020}
\def\mfnazov{Mobilná aplikácia na nájdenie optimálnej trasy v mhd z reálnych dát}
\def\mftyp{Diplomová práca}
\def\mfautor{Bc. Gabriela Slaninková}
\def\mfskolitel{doc. RNDr. Milan Ftáčnik, CSc.}

%ak mate konzultanta, odkomentujte aj jeho meno na titulnom liste
\def\mfkonzultant{Mgr. Ľubor Illek }  

\def\mfmiesto{Bratislava, \mfrok}

%aj cislo odboru je povinne a je podla studijneho odboru autora prace
\def\mfodbor{2511 Aplikovaná informatika} 
\def\program{ Aplikovaná informatika }
\def\mfpracovisko{ Katedra aplikovanej informatiky }

\begin{document}     
\frontmatter


% -------------------
% --- Obalka ------
% -------------------
\thispagestyle{empty}

\begin{center}
\sc\large
Univerzita Komenského v Bratislave\\
Fakulta matematiky, fyziky a informatiky

\vfill

{\LARGE\mfnazov}\\
\mftyp
\end{center}

\vfill

{\sc\large 
\noindent \mfrok\\
\mfautor
}

\eject % EOP i
% --- koniec obalky ----

% -------------------
% --- Titulný list
% -------------------

\thispagestyle{empty}
\noindent

\begin{center}
\sc  
\large
Univerzita Komenského v Bratislave\\
Fakulta matematiky, fyziky a informatiky

\vfill

{\LARGE\mfnazov}\\
\mftyp
\end{center}

\vfill

\noindent
\begin{tabular}{ll}
Študijný program: & \program \\
Študijný odbor: & \mfodbor \\
Školiace pracovisko: & \mfpracovisko \\
Školiteľ: & \mfskolitel \\
Konzultant: & \mfkonzultant \\
\end{tabular}

\vfill


\noindent \mfmiesto\\
\mfautor

\eject % EOP i


% --- Koniec titulnej strany


% -------------------
% --- Zadanie z AIS
% -------------------
% v tlačenej verzii s podpismi zainteresovaných osôb.
% v elektronickej verzii sa zverejňuje zadanie bez podpisov

\newpage 
\thispagestyle{empty}
\hspace{-0.7cm}\includegraphics[width=1.1\textwidth]{images/zada}

% --- Koniec zadania

\frontmatter

% -------------------
%   Poďakovanie - nepovinné
% -------------------
\setcounter{page}{3}
\newpage 
~

\vfill
{\bf Poďakovanie:}
\\ \\
Chcela by som sa poďakovať vedúcemu mojej diplomovej práce doc. RNDr. Milanovi Ftáčnikovi, CSc. za jeho cenné rady, pripomienky, ústretový prístup a usmerňovanie pri
tvorbe diplomovej práce. Moja vďaka patrí aj Mgr. Ľuborovi Illekovi za jeho čas a rady. Za poskytnutie dát ďakujem Dopravnému podniku Bratislava.




% --- Koniec poďakovania

% -------------------
%   Abstrakt - Slovensky
% -------------------
\newpage 
\section*{Abstrakt}
Diplomová práca popisuje vývoj progresívnej webovej aplikácie na vyhľadávanie ciest v bratislavskej mestskej hromadnej doprave. Aplikácia pri vyhľadávaní ciest prihliada na aktuálne meškanie vozidiel. Nehľadá len jednu najkratšiu cestu, ale ponúka cestujúcim alternatívne optimálne cesty a možnosť navolenia rôznych preferencií. Medzi jej ďalšie funkcionality patrí vyhľadávanie cesty z aktuálnej polohy cestujúceho, ako aj možnosť voľby zastávky priamo z mapy. Súčasťou tejto aplikácie je taktiež zobrazenie všetkých liniek a ich trás.

Kľúčovou časťou diplomovej práce je výber a implementácia vhodného vyhľadávacieho algoritmu, ktorý sa dokáže vysporiadať s časovo závislými dátami, alternatívnymi cestami a multimodalitou hromadnej dopravy. 
Našou voľbou je negrafový RAPTOR algoritmus, ktorý sme prispôsobili tak, aby vedel pracovať s dynamickými dátami a aby bol schopný spracovať preferencie vyhľadávania.

\paragraph*{Kľúčové slová:} 
RAPTOR algoritmus, hľadanie optimálnej cesty v mestskej hromadnej doprave, multimodalita, alternatívne cesty, progresívna webová aplikácia
% --- Koniec Abstrakt - Slovensky


% -------------------
% --- Abstrakt - Anglicky 
% -------------------
\newpage 
\section*{Abstract}
Diploma thesis describes development of progressive web application for journey planning in public transport in Bratislava. Application take into consideration actual delay of vehicles while searching journeys. Our application is not searching for just one shortest path, but it is providing alternative optimal paths as well as possibility to choose among variety of searching preferences. There are other functionalities like searching paths from passenger current location or possibility to choose the stop from a map. This application also offers option to list all routes and their stop sequences.

Key part of diploma thesis is also the selection and implementation of applicable searching algorithm that can deal with time dependant data, searching alternative paths and multimodality of public transport.  
Our choice is non-graph RAPTOR algorithm, which we modified to handle dynamic data and searching preferences.

\paragraph*{Key words:}
RAPTOR algorithm, optimal path finding in the city public transport, multimodality, alternative paths, progressive web application
% --- Koniec Abstrakt - Anglicky

% -------------------
% --- Obsah
% -------------------

\newpage 

\tableofcontents

% ---  Koniec Obsahu

% -------------------
% --- Zoznamy tabuliek, obrázkov - nepovinne
% -------------------

\newpage 

\listoffigures
\listoftables

% ---  Koniec Zoznamov

\mainmatter


\input uvod.tex 

\input vychodiska.tex

% \input specifikacia.tex

\input navrh.tex

\input implementacia.tex

\input test.tex

\input zaver.tex

% -------------------
% --- Bibliografia
% -------------------


\newpage	

\backmatter

\thispagestyle{empty}
\nocite{*}
\clearpage

\bibliographystyle{plain}
\bibliography{literatura} 

%---koniec Referencii

% -------------------
%--- Prilohy---
% -------------------

%Nepovinná časť prílohy obsahuje materiály, ktoré neboli zaradené priamo  do textu. Každá príloha sa začína na novej strane.
%Zoznam príloh je súčasťou obsahu.
%
%\addcontentsline{toc}{chapter}{Appendix A}
%\input AppendixA.tex
%
%\addcontentsline{toc}{chapter}{Appendix B}
%\input AppendixB.tex

\end{document}






