\phantomsection
\chapter*{Úvod} % chapter* je necislovana kapitola
\addcontentsline{toc}{chapter}{Úvod} % rucne pridanie do obsahu
\markboth{Úvod}{Úvod} % vyriesenie hlaviciek

Hlavné mesto je obrazom celého štátu a infraštruktúra je jeho neoddeliteľnou súčasťou. Mestská hromadná doprava je nevyhnutným spôsobom prepravy pre každé mesto, pre Bratislavu obzvlášť, keďže naše hlavné mesto nemá vybudované mestské podzemné dráhy.

MHD pomáha nielen ľudom, ktorí nevlastnia osobné vozidlá, ale aj zdravotne znevýhodneným a účastníkom cestovného ruchu. Navyše minimalizuje počet aut v meste, čím odbremeňuje mesto od dopravných zápch. Prispieva k ekologickej preprave v meste a v niektorých prípadoch môže byť mestská hromadná doprava najrýchlejším a najekonomickejším spôsobom prepravy.
Aby bolo cestovanie verejnou dopravou komfortné a cestujúci viac motivovaní, kľúčovú rolu zohráva presnosť spojov, vysoká frekvencia spojov, hustota rozsadenia zastávok, minimálna nutnosť peších presunov medzi zastávkami, pohodlnosť cestovania a mnohé iné faktory. 

Študent informatiky je schopný prispieť k zefektívneniu MHD vytvorením aplikácie, ktorá bude ponúkať čo najpresnejšie vyhľadávanie spojov, okolitých zastávok a prispeje k celkovému sprehľadneniu hromadnej dopravy.

Aktuálne sú dostupné mnohé aplikácie, ktoré poskytujú spomínané funkcionality. Okrem možnosti vyhľadávania spojov ponúkajú aj priamu kúpu cestovného lístka, sledovanie pohybu vozidiel v reálnom čase, ako aj informáciu o možnosti prestupu na prímestskú dopravu. Avšak vyhľadávanie je realizované nad statickými cestovnými poriadkami. 

Dopravný podnik Bratislava disponuje informáciami o aktuálnych polohách vozidiel, z čoho vie určiť meškanie jednotlivých spojov. Tieto informácie sú zobrazené na elektronických tabuliach na vybraných zastávkach. Prečo nevyužiť tieto dáta na spresnenie vyhľadávania spojení?

Rozhodli sme sa vytvoriť aplikáciu, ktorá bude pri vyhľadávaní cesty prihliadať na aktuálnu dopravnú situáciu. Zároveň chceme vytvoriť túto aplikácie pre široké spektrum cestujúcich. Vieme, že existujú cestujúci, ktorí preferujú minimálny počet prestupov, iní hľadajú najrýchlejšiu cestu do cieľa a sú aj takí, ktorí potrebujú na svoj presun nízkopodlažné vozidlá. Preto táto aplikácia bude poskytovať alternatívne optimálne cesty.


 
