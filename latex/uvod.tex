\phantomsection
\chapter*{Úvod} % chapter* je necislovana kapitola
\addcontentsline{toc}{chapter}{Úvod} % rucne pridanie do obsahu
\markboth{Úvod}{Úvod} % vyriesenie hlaviciek

Dopravná infraštruktúra je neoddeliteľnou súčasťou hlavného mesta najmä vtedy, ak mesto nemá vybudovanú podzemnú dráhu. MHD slúži všetkým ľudom, najmä tým, ktorí nevlastnia osobné vozidlá, sú zdravotne znevýhodnení, ako aj účastníkom cestovného ruchu. Navyše minimalizuje počet vozidiel v meste, čím odbremeňuje mesto od dopravných zápch. Prispieva k ekologickej preprave v meste a v niektorých prípadoch môže byť mestská hromadná doprava najrýchlejším a najekonomickejším spôsobom prepravy cestujúcich.
Aby bolo cestovanie verejnou dopravou komfortné a cestujúci viac motivovaní využívať tento spôsob prepravy, kľúčovú rolu zohráva presnosť spojov, ich vysoká frekvencia, hustota rozmiestnenia zastávok, minimálna nutnosť peších presunov medzi zastávkami, pohodlnosť cestovania a mnohé iné faktory. 

Študent informatiky je schopný prispieť k zefektívneniu MHD vytvorením aplikácie, ktorá bude ponúkať čo najpresnejšie vyhľadávanie ciest a okolitých zastávok, aby tak prispel k celkovému sprehľadneniu mestskej hromadnej dopravy.

Aktuálne sú dostupné mnohé aplikácie, ktoré poskytujú spomínané funkcionality. Okrem možnosti vyhľadávania ciest ponúkajú aj priamu kúpu cestovného lístka, sledovanie pohybu vozidiel v reálnom čase, ako aj informáciu o možnosti prestupu na prímestskú dopravu. Avšak vyhľadávanie je realizované nad statickými cestovnými poriadkami. 

Dopravný podnik Bratislava disponuje informáciami o aktuálnych polohách vozidiel, z čoho vie následne určiť meškanie jednotlivých spojov. Tieto informácie sú zobrazené na elektronických tabuliach na vybraných zastávkach. Prečo teda nevyužiť tieto dáta na spresnenie vyhľadávania ciest? 

Rozhodli sme sa vytvoriť aplikáciu, ktorá bude pri vyhľadávaní ciest prihliadať na aktuálnu dopravnú situáciu. Zároveň chceme vytvoriť túto aplikáciu pre široké spektrum cestujúcich. Vieme, že existujú cestujúci, ktorí preferujú minimálny počet prestupov, iní hľadajú najrýchlejšiu cestu do cieľa a sú aj takí, ktorí potrebujú na svoj presun nízkopodlažné vozidlá. Preto táto aplikácia bude poskytovať aj alternatívne optimálne cesty.

Práca popisuje priebeh vývoja aplikácie a je rozdelená do štyroch kapitol. 
V prvej kapitole sú zhrnuté poznatky, ktoré sme nadobudli štúdiom zaujímavých vedeckých článkov, 
ako aj skúmaním funkcií už existujúcich aplikácií na vyhľadávanie ciest v bratislavskej mestskej hromadnej doprave. V literatúre sa vyhľadávanie ciest často spája s Dijkstrovým algoritmom.  
Spomíname aj algoritmus A* spolu s jeho rôznymi optimalizáciami a modifikáciami. Jedna z modifikácií sa dokonca zaoberá časovými dátami a počíta s multimodalitou dopravnej siete. Pri grafových algoritmoch sú často spomínané aj minimalizácie prehľadávaného priestoru, ktoré by mohli zrýchliť beh algoritmu. Zmieňujeme sa aj o takej literatúre, ktorá by nám mohla pomôcť s vyhľadávaním alternatívnych ciest. Spomedzi všetkých algoritmov na vyhľadávanie ciest nás napokon najviac zaujal negrafový algoritmus RAPTOR, čo je skratkou pre Round bAsed Public Transit Optimal Router. Ako vyplýva už z názvu, je určený na vyhľadávanie ciest v sieti verejnej dopravy. V tejto kapitole popisujeme jeho základnú verziu, ale aj vylepšenú verziu, ktorá sa zaoberá vyhľadávaním alternatívnych ciest. Výhodou algoritmu je, že je prispôsobiteľný tak, aby spracovával aj dynamické dáta. 

Druhou kapitolou je kapitola Návrh, v ktorej vysvetľujeme, ako sme sa rozhodli vytvoriť našu aplikáciu z pohľadu funkcií, vyhľadávacieho algoritmu, dát a architektúry. Popisujeme funkcie, ktoré bude aplikácia cestujúcim ponúkať, pričom sme sa inšpirovali existujúcimi aplikáciami a vlastnými skúsenosťami. V tejto kapitole objasňujeme výber algoritmu na vyhľadávanie ciest. Vybraným algoritmom je RAPTOR algoritmus, ktorý modifikujeme, aby sme dosiahli požadované funkcie aplikácie. Ďalej po podrobnej analýze približujeme návrh spracovania dát a popisujeme dáta statických cestovných poriadkov uložených vo formáte GTFS. Dopravný podnik Bratislava nám poskytol aj dáta o meškaní vozidiel. Opisujeme, ako ukladáme dáta do databázy a do dátovej štruktúry pre algoritmus. V neposlednom rade popisujeme aj návrh architektúry systému.

V kapitole Implementácia približujeme technológie použité pri vývoji aplikácie. Spomíname riešenie problémov vzniknutých pri importovaní a spracovávaní statických dát a dát o meškaní. Popisujeme, akým spôsobom je implementovaná dátová štruktúra a ako do nej ukladáme dáta o meškaní vozidiel. Pre správne fungovanie aplikácie bolo potrebné simulovať čas v minulosti, keďže poskytnuté dáta mali platnosť v roku 2018. Ďalej ukazujeme implementáciu niektorých kľúčových častí RAPTOR algoritmu. Na ukážku poskytujeme časti kódu, ktorými sme obohatili algoritmus, aby prihliadal na preferencie zadané používateľom. Približujeme taktiež problémy, ktoré vznikali pri implementácii na klientskej strane.

V poslednej kapitole spomíname testovacie dáta, ktoré sme vytvorili na overenie správnosti vyhľadávacieho algoritmu. Na ukážkach vyhľadaných ciest približujeme, ako sa algoritmus postupne vyvíjal a podľa toho sa menil zoznam vyhľadaných ciest. 
