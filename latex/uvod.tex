\phantomsection
\chapter*{Úvod} % chapter* je necislovana kapitola
\addcontentsline{toc}{chapter}{Úvod} % rucne pridanie do obsahu
\markboth{Úvod}{Úvod} % vyriesenie hlaviciek

Dopravná infraštruktúra je neoddeliteľnou súčasťou mobility mesta. Mestská hromadná doprava slúži všetkým ľudom žijúcim v meste, tým čo dochádzajú do mesta za rôznym cieľom, ako aj účastníkom cestovného ruchu. Navyše minimalizuje počet vozidiel v meste, čím odbremeňuje mesto od dopravných zápch. Prispieva k ekologickej preprave v meste a v niektorých prípadoch môže byť mestská hromadná doprava najrýchlejším a najekonomickejším spôsobom prepravy cestujúcich. Aby bolo cestovanie verejnou dopravou komfortné a cestujúci viac motivovaní využívať tento spôsob prepravy, kľúčovú rolu zohráva presnosť spojov, ich vysoká frekvencia, hustota rozmiestnenia zastávok, minimálna nutnosť peších presunov medzi zastávkami, pohodlnosť cestovania, informovanosť o optimálnych cestách a mnohé iné faktory.

Cieľom práce je prispieť k zefektívneniu MHD vytvorením aplikácie,
ktorá bude ponúkať čo najpresnejšie vyhľadávanie ciest a okolitých zastávok, aby tak prispel k celkovému sprehľadneniu mestskej hromadnej dopravy.

Aktuálne sú dostupné mnohé aplikácie, ktoré poskytujú spomínané funkcionality. Okrem možnosti vyhľadávania ciest ponúkajú aj priamu kúpu cestovného lístka, sledovanie pohybu vozidiel v reálnom čase, ako aj informáciu o možnosti prestupu na prímestskú dopravu. Avšak vyhľadávanie je realizované nad statickými cestovnými poriadkami. Dopravný podnik Bratislava disponuje informáciami o aktuálnych polohách vozidiel, z čoho vie následne určiť meškanie jednotlivých spojov. Tieto informácie sú zobrazené na elektronických tabuliach na vybraných zastávkach.

Na základe zadania sme sa rozhodli vytvoriť aplikáciu, ktorá bude prihliadať na aktuálnu dopravnú situáciu a bude slúžiť pre široké spektrum cestujúcich. Vieme, že existujú cestujúci, ktorí preferujú minimálny počet prestupov, iní hľadajú najrýchlejšiu cestu do cieľa a sú aj takí, ktorí potrebujú na svoj presun nízkopodlažné vozidlá. Preto táto aplikácia bude poskytovať aj alternatívne optimálne cesty.

Práca popisuje priebeh vývoja aplikácie a je rozdelená do štyroch kapitol. 

V prvej kapitole sú zhrnuté poznatky, ktoré sme nadobudli štúdiom zaujímavých vedeckých článkov, ako aj skúmaním funkcií už existujúcich aplikácií na vyhľadávanie ciest v bratislavskej mestskej hromadnej doprave. Zameriavame sa na literatúru, ktorá spomína vyhľadávanie ciest v hromadnej doprave a ktorá rieši niektoré z problémov: multimodalita dopravy, dynamické dáta alebo alternatívne cesty. Existuje veľa grafových riešení, ale natrafili sme aj na negrafové riešenie vyhľadávania ciest. 

Druhou kapitolou je kapitola Návrh, v ktorej vysvetľujeme, ako sme sa rozhodli vytvoriť našu aplikáciu z pohľadu funkcií, vyhľadávacieho algoritmu, dát a architektúry. Objasňujeme výber RAPTOR algoritmu a navrhujeme jeho modifikácie na dosiahnutie požadovaných funkcií. Opisujeme dáta získané od Dopravného podniku Bratislava, ako aj návrh ich ukladania do databázy a dátovej štruktúry pre algoritmus. Navrhujeme ako sa vysporiadať s dynamickými dátami a približujeme návrh klient-server architektúry.

V kapitole Implementácia popisujeme technológie použité pri vývoji aplikácie. Čo sa týka serverovej strany, spomíname implementáciu vyhľadávacieho algoritmu, dátovej štruktúry a importovania dát. Ďalej opisujeme ako prebiehal vývoj vybraných častí na klientskej strane. Keďže platnosť cestovných poriadkov a dát o meškaní vozidiel je z minulosti, popíšeme ako sme simulovali čas v aplikácii. 

V poslednej kapitole spomíname testovacie dáta, ktoré sme vytvorili na prvotné overenie správnosti algoritmu. Konkrétnymi ukážkami demonštrujeme schopnosť algoritmu vyhľadávať cesty na reálnych dátach od Dopravného podniku Bratislava. Ukážeme, ako algoritmus pri vyhľadávaní prihliada na meškanie vozidiel alebo na zadané používateľské preferencie. V neposlednom rade predvedieme, ako algoritmus hľadá alternatívne optimálne cesty. 


