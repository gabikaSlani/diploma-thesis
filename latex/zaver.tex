\chapter*{Záver}  % chapter* je necislovana kapitola
\addcontentsline{toc}{chapter}{Záver} % rucne pridanie do obsahu
\markboth{Záver}{Záver} % vyriesenie hlaviciek

Podarilo sa nám navrhnúť a naimplementovať aplikáciu na vyhľadávanie optimálnych ciest v MHD Bratislava. Dominantnou funkciou aplikácie je, že prihliada na aktuálnu dopravnú situáciu a pri vyhľadávaní ciest zohľadňuje aj meškanie vozidiel. Keďže cestujúci majú rôzne preferencie a optimálna cesta môže byť pre každého iná, aplikácia neponúka používateľovi len jednu cestu, ale viacero alternatívnych optimálnych ciest. Používateľ môže zvoliť jeho aktuálnu polohu ako začiatočný bod. V prípade, že používateľ pozná názvy zastávok, môže ich vybrať zo zoznamu ako začiatočnú a cieľovú zastávku, prípadne si ich môže zvoliť z mapy. Okrem iného aplikácia ponúka aj zobrazenie všetkých liniek a ich trás. 

Pred samotným návrhom a vývojom aplikácie sme študovali literatúru, ktorá sa venovala riešeniu rôznych otázok vznikajúcich pri hľadaní ciest vo verejnej doprave. Kľúčovým problémom bol najmä výber algoritmu na vyhľadávanie ciest. Vo fáze štúdia odbornej literatúry sme sa stretli najmä s grafovými riešeniami hľadania ciest. Našli sme aj rôzne modifikácie a optimalizácie týchto algoritmov. My sme sa však zaujímali najmä o také algoritmy, ktoré sa dokážu vysporiadať s časovo závislými dátami, rôznymi typmi dopravných prostriedkov (módmi) a alternatívnymi cestami. Pri tvorení požiadaviek na aplikáciu sme sa inšpirovali podobnými existujúcimi aplikáciami, ktoré okrem vyhľadávania spojov ponúkajú rôzne iné funkcionality. O tom sa zmieňujeme v kapitole Východiská.

Napokon sme sa rozhodli pre algoritmus, ktorý využíva fakt, že mestská doprava jazdí po vopred určených trasách. Týmto algoritmom je RAPTOR algoritmus, ktorého výhodou je, že je pomerne jednoducho prispôsobiteľný. Navrhli sme zmeny ako upraviť algoritmus, aby vedel pracovať s dátami poskytnutými Dopravným podnikom Bratislava. Ďalej sme navrhli dátovú štruktúru pre algoritmus, ktorá udržuje dáta statických cestovných poriadkov ako aj dáta o meškaní vozidiel. Bolo nutné navrhnúť ďalšie zmeny v algoritme, aby zohľadňoval pri vyhľadávaní zadané používateľské preferencie. Navrhli sme filtre, ktoré minimalizujú množinu všetkých ciest na množinu optimálnych ciest. 

Vo fáze implementácie sme natrafili na niekoľko problémov s dátami, keďže nevyhovovali forme, ktoré RAPTOR algoritmus vyžadoval. Museli sme sa vysporiadať s veľkosťou dátovej štruktúry pre algoritmus, ako aj s neaktuálnymi dátami. Vytvorili sme simulátor času, keďže dáta sú z roku 2018 a my potrebujeme simulovať vyhľadávanie v aktuálny deň. Na klientskej strane vznikol problém so zobrazením zastávok na mape tak, aby zobrazenie bolo čo najviac používateľsky prívetivé.

Aplikáciu sme testovali v jednotlivých fázach vývoja. Algoritmus sme stále obohacovali o nové vylepšenia. Začali sme vyhľadávaním nad statickými cestovnými poriadkami, pokračovali sme prispôsobením algoritmu a dátovej štruktúry, aby prihliadal na meškania spojov a poskytoval cesty zodpovedajúce aktuálnej dopravnej situácii. Ďalej sme vytvorili a aplikovali filtre na cesty tak, že výsledkom sú vždy optimálne cesty. V poslednom kroku sme prispôsobili algoritmus, aby pri vyhľadávaní vedel zohľadniť používateľské kritériá. Pre potreby testovania, ale aj vylepšovania algoritmu sme si vytvorili testovacie dáta. Správnosť vyhľadaných ciest tak mohla byť kontrolovaná. 

Aplikácia má niekoľko častí, ktoré by bolo možné v budúcnosti vylepšiť. 

Počas implementácie sme objavili dáta, ktoré sa nám nepodarilo zanalyzovať. Išlo o dáta, ktoré predstavujú časy odchodu zo zastávok po 23:59 hodine. Tieto dáta sme odignorovali a z toho dôvodu nemusí správne fungovať vyhľadávanie na prelome dní. 

Ďalšou možnosťou vylepšenia by bolo zvážiť, v akej fáze vyhľadávania ciest, je najvhodnejšie filtrovať vyhovujúce cesty na optimálne. V tomto riešení zapracovávame filtre na vyhľadané cesty po zbehnutí algoritmu. Stálo by za zváženie, či nie je efektívnejšie ich filtrovať už počas behu algoritmu.

V neposlednom rade by mohlo byť vyhľadávanie paralelizované na viacerých CPU jadrách. Linky v dátovej štruktúre sú zapísané nezávisle a pri vyhľadávaní môže každé jadro spracovávať inú podmnožinu liniek zvlášť.
