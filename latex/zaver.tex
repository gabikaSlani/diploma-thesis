\chapter*{Záver}  % chapter* je necislovana kapitola
\addcontentsline{toc}{chapter}{Záver} % rucne pridanie do obsahu
\markboth{Záver}{Záver} % vyriesenie hlaviciek

Podarilo sa nám navrhnúť a naimplementovať aplikáciu na vyhľadávanie optimálnych ciest v mestskej hromadnej doprave Bratislava. Dominantnou funkciou aplikácie je, že prihliada na aktuálnu dopravnú situáciu a pri vyhľadávaní ciest zohľadňuje aj meškanie vozidiel. Vyhľadávací algoritmus beží na serverovej strane, pričom operuje na stále aktuálnej dátovej štruktúre. Na vyhľadávanie sme použili negrafový RAPTOR algoritmus, ktorý ľahko zvláda dynamickosť dát, je prispôsobivý a nepotrebuje vytvárať model pred každým spustením algoritmu. 

Optimálna cesta môže byť pre každého cestujúceho iná. Najrýchlejšia cesta nemusí byť vždy tá, ktorá cestujúcim vyhovuje najviac. Naša aplikácia ponúka viacero optimálnych ciest, pričom cesty sa líšia v počte prestupov a obsahujú vždy aj najrýchlejšiu cestu.

Pri vyhľadávaní ciest používateľ zadáva požadovaný začiatočný, konečný bod cesty a dátum a čas začiatku vyhľadávania, pričom predvolený je aktuálny dátum a čas. Aplikácia ponúka aj rozšírené vyhľadávacie parametre. Používateľ má teda možnosť nastavenia maximálneho počtu prestupov, minimálneho času na prestup medzi jazdami, maximálneho času pešieho presunu, ako aj vyhľadanie len nízkopodlažných spojov. V prípade, že používateľ upraví rozšírené vyhľadávacie parametre, algoritmus prispôsobí vyhľadávanie zadaným parametrom. Zmenené hodnoty preferencií sa ukladajú do zariadenia a sú použité aj pri nasledujúcich vyhľadávaniach. Používateľ môže vidieť v histórii vyhľadávania 5 posledných záznamov (začiatočná a konečná zastávka). 

Aplikácia poskytuje nielen tie optimálne cesty, ktoré začínajú najbližšie od zadaného času a dátumu, ale aj nasledujúce v chronologickom časovom slede. Vždy zobrazí aspoň 5 ciest na jednu stránku. Následne môže používateľ zvoliť voľbu pre vyhľadanie ďalších nasledujúcich ciest.

Používateľ má možnosť zvoliť jeho aktuálnu polohu ako začiatočný bod. Konečný bod môže byť len zastávka. V prípade, že používateľ pozná názvy zastávok, môže ich vybrať zo zoznamu ako začiatočnú a konečnú zastávku. Ak pomenovanie jednotlivých zastávok nepozná, môže si ich zvoliť priamo z mapy. Okrem iného, aplikácia ponúka aj zobrazenie všetkých liniek a ich trás. Trasu linky si môže používateľ zobraziť na mape. 

Momentálne má aplikácia v sebe naprogramovaný simulátor času, aby sme mohli vyhľadávať cesty v reálnom čase a prihliadať tak na meškania vozidiel, aj keď máme k dispozícii iba dáta z minulosti. 

Testovanie prebiehalo na reálnej sade dát od Dopravného podniku Bratislava. Dáta statických cestovných poriadkov, ako aj dáta o meškaní vozidiel sú z roku 2018. Keďže sme nevedeli vyhodnocovať správnosť vyhľadaných ciest, najskôr sme si vytvorili zmenšenú sadu dát. Správnosť vyhľadaných ciest tak mohla byť kontrolovaná manuálne už vo fáze vývoja. 

Na reálnej sade dát sme testovali, ako sa množina vyhľadaných ciest mení pri zmene používateľských preferencií. Ukázali sme, že vyhľadané cesty sa menia aj podľa toho, či prihliadame na meškanie vozidiel. Cesty sú optimálne podľa takých pravidiel, ako sme určili pri návrhu aplikácie. 

Aplikácia má niekoľko častí, ktoré by bolo možné v budúcnosti vylepšiť. 

Počas implementácie sme objavili dáta, ktoré sa nám nepodarilo zanalyzovať. Išlo o dáta, ktoré predstavujú časy odchodu zo zastávok po 23:59 hodine. Tieto dáta sme odignorovali a z toho dôvodu nemusí správne fungovať vyhľadávanie na prelome dní. 

Ďalšou možnosťou vylepšenia by bolo zvážiť, v akej fáze vyhľadávania ciest je najvhodnejšie filtrovať vyhovujúce cesty na optimálne. Aby sme používateľovi ponúkli alternatívne cesty, algoritmus hľadá všetky cesty zo začiatočnej zastávky do konečnej zastávky. Následne na tieto cesty aplikujeme filtre a získavame množinu optimálnych ciest. Stálo by za zváženie a overenie, či nie je efektívnejšie ich filtrovať už počas behu algoritmu.

V neposlednom rade by mohlo byť vyhľadávanie paralelizované na viacerých CPU jadrách. Linky v dátovej štruktúre sú zapísané nezávisle a pri vyhľadávaní môže každé jadro spracovávať inú podmnožinu liniek zvlášť.

